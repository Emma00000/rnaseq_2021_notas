% Options for packages loaded elsewhere
\PassOptionsToPackage{unicode}{hyperref}
\PassOptionsToPackage{hyphens}{url}
%
\documentclass[
]{article}
\usepackage{lmodern}
\usepackage{amssymb,amsmath}
\usepackage{ifxetex,ifluatex}
\ifnum 0\ifxetex 1\fi\ifluatex 1\fi=0 % if pdftex
  \usepackage[T1]{fontenc}
  \usepackage[utf8]{inputenc}
  \usepackage{textcomp} % provide euro and other symbols
\else % if luatex or xetex
  \usepackage{unicode-math}
  \defaultfontfeatures{Scale=MatchLowercase}
  \defaultfontfeatures[\rmfamily]{Ligatures=TeX,Scale=1}
\fi
% Use upquote if available, for straight quotes in verbatim environments
\IfFileExists{upquote.sty}{\usepackage{upquote}}{}
\IfFileExists{microtype.sty}{% use microtype if available
  \usepackage[]{microtype}
  \UseMicrotypeSet[protrusion]{basicmath} % disable protrusion for tt fonts
}{}
\makeatletter
\@ifundefined{KOMAClassName}{% if non-KOMA class
  \IfFileExists{parskip.sty}{%
    \usepackage{parskip}
  }{% else
    \setlength{\parindent}{0pt}
    \setlength{\parskip}{6pt plus 2pt minus 1pt}}
}{% if KOMA class
  \KOMAoptions{parskip=half}}
\makeatother
\usepackage{xcolor}
\IfFileExists{xurl.sty}{\usepackage{xurl}}{} % add URL line breaks if available
\IfFileExists{bookmark.sty}{\usepackage{bookmark}}{\usepackage{hyperref}}
\hypersetup{
  pdftitle={01\_apuntes\_rnaseq},
  pdfauthor={Emmanuel},
  hidelinks,
  pdfcreator={LaTeX via pandoc}}
\urlstyle{same} % disable monospaced font for URLs
\usepackage[margin=1in]{geometry}
\usepackage{graphicx,grffile}
\makeatletter
\def\maxwidth{\ifdim\Gin@nat@width>\linewidth\linewidth\else\Gin@nat@width\fi}
\def\maxheight{\ifdim\Gin@nat@height>\textheight\textheight\else\Gin@nat@height\fi}
\makeatother
% Scale images if necessary, so that they will not overflow the page
% margins by default, and it is still possible to overwrite the defaults
% using explicit options in \includegraphics[width, height, ...]{}
\setkeys{Gin}{width=\maxwidth,height=\maxheight,keepaspectratio}
% Set default figure placement to htbp
\makeatletter
\def\fps@figure{htbp}
\makeatother
\setlength{\emergencystretch}{3em} % prevent overfull lines
\providecommand{\tightlist}{%
  \setlength{\itemsep}{0pt}\setlength{\parskip}{0pt}}
\setcounter{secnumdepth}{-\maxdimen} % remove section numbering

\title{01\_apuntes\_rnaseq}
\author{Emmanuel}
\date{23/2/2021}

\begin{document}
\maketitle

\hypertarget{rstudio-y-github}{%
\subsection{Rstudio y Github}\label{rstudio-y-github}}

\textbf{Github} te permite compartir códigos y tener un control de tus
códigos, permite varios lenguajes de programación y varios tipos de
documentos como texto plano, pdfs, imágenes, entre otros.

En \textbf{Rstudio} es una forma más gráfica de manejar R y te permite
realizar varias cosas de forma más rápida que con solo R. También te
permite crear proyectos, documentos en R, en Rmarkdown, etc. También
facilita la conversión a un html o un pdf de tus códigos, así como las
gráficas de estos a imágenes que puedes almacenar. Otra ventaja de
Rstudio es que puedes crear repositorios locales de git y después
subirlos a github.

\textbf{Error al crear un repositorio y alternativa}

\begin{verbatim}
gitcreds::gitcreds_set() #Sirve para tener las credenciales adecuadas para manejar git.
#Hay ocasiones en las que no funciona bien, para lo cual se utiliza una forma no recomendada
usethis::edit_r_environ()
#Dentro del documento que se abre por la línea anterior escribe: 
#GITHUB_PAT=TU_CLAVE_DE_40_LETRAS
#
#IMPORTANTE dejar el salto de línea.
\end{verbatim}

En caso de que el boton de git de la parte superior derecha no aparezca
después del \emph{usethis::use\_git()} reiniciar R en \emph{Session} de
la parte superior de las herramientas y \emph{Restart R}. En caso de no
tenerlo después del procedimiento anterior verificar que tengas git
instalado y que estes dentro del proyecto/repositorio.

\textbf{Códigos alternos en caso de no tener el boton de git:}

\begin{verbatim}
#Primero se debe revisar que archivos con modificaciones hay que no se han subido a github.
gert::git_status()
#En el caso de la última columna tendrá un valor booleano: 
#"true" para los que se han añadido pero no se han comentado.
#"False" para los archivos que no se han añadido ni comentado.

#Para los archivos no añadidos se utiliza:
gert::git_add("file.example") #Pueden ser varios archivos, se pasan en un vector con sus nombres.

#Para los archivos sin comentar:
gert::git_commit("comentario de la modificación")
#o si son varios:
gert::git_commit_all("comentario de las modificaciones")

#Y por último para subir las modificaciones a github:
gert::git_push()
\end{verbatim}

\hypertarget{ejercicio-postcards}{%
\subsection{Ejercicio postcards}\label{ejercicio-postcards}}

\begin{verbatim}
#Creacion de nuestra página web, se utiliza nuestro usuario de github.
usethis::create_project("Emma00000.github.io")

## Configura Git y GitHub
usethis::use_git()
usethis::use_github()
#Se utiliza uno de los templados que tiene postcards.
postcards::create_postcard(template = "onofre")
#Se edita para que tenga nuestra información, se hace el commit y se sube a github.
\end{verbatim}

\hypertarget{bioconductor}{%
\subsection{Bioconductor}\label{bioconductor}}

Es un repositorio de paquetes de R que está enfocado al análisis de
datos genómicos. Tienen varios tipos de paquetes: Software:
\url{http://bioconductor.org/packages/release/bioc/} Annotation:
\url{http://bioconductor.org/packages/release/data/annotation/}
Experiment Data:
\url{http://bioconductor.org/packages/release/data/experiment/}
Workflows: \url{http://bioconductor.org/packages/release/workflows/}
Todos sus paquetes están descritos en un html asignado para cada uno
siguiendo la lógica: \url{https://bioconductor.org/packages/} . Esta
descripción que tiene cada paquete cuenta con varias cosas para
considerar usarlo o no, las preguntas que se le han hecho a los
desarrolladores y sus respuestas, un puntaje que se le asigna, la
documentación que tiene sus códigos, el tiempo que tienen en haberse
creado, entre otros.

\hypertarget{ejercicio-grupal}{%
\subsection{Ejercicio grupal}\label{ejercicio-grupal}}

\textbf{Velociraptor:} El paquete de velociraptor proporciona un
``wrapper'' para cálculos de velocidad de ARN en datos de ARN-seq de una
sola célula. Se utilizan paquetes como basilisk y zellkonverter para
convertir los datos de SingleCellExperiment (R) y AnnData (Python). Su
documentación es aceptable, clara, siento que le falta en algunos
puntos. Como es su primera versión es aceptable; no tiene preguntas en
su html, pero en su github si tiene y se han contestado casi todas de
una buena forma, exceptuando una muy reciente.

URL
-\url{http://bioconductor.org/packages/release/bioc/html/velociraptor.html}

\textbf{ChromSCape} Es un paquete que perfila el paisaje de cromatina
para Single Cell, analiza conjuntos de datos de epigenética y cubre
pasos como el preprocesamiento, filtrado, batch correction,
visualización, clustering análisis diferencial y análisis de conjuntos
de genes. La documentación no es muy extensa pero es clara y hasta el
momento no hay preguntas de usuarios. Hasta ahora funciona en todas las
plataformas. El repositorio tiene orden, aunque la documentación del
código no es tan explícita.

\textbf{ToxicoGx} El paquete permite analizar datos toxico genomicos,
tanto para visualización y analisis de estos datos. - En las pruebas en
las distintas plataformas no se han encontrados errores - Hay una
documentación adecuada y ejemplos de uso - No tiene preguntas hasta
ahora La URL al codigo no esta clara Link:
\url{https://bioconductor.org/packages/release/bioc/html/ToxicoGx.html}

\textbf{Nebulosa} Permite una visualización de datos de experimentos de
una sola celula. - Presenta errores en todas las plataformas -
Documentación bastante breve - No tiene preguntas hasta ahora Link:
\url{https://bioconductor.org/packages/release/bioc/html/Nebulosa.html}

\end{document}
